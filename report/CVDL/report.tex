\documentclass[a4paper,12pt]{article}
\usepackage{CJKutf8}
\usepackage{indentfirst}
\usepackage{graphicx}
\usepackage{amsmath}
\usepackage{longtable}
\usepackage{fancyhdr}
\usepackage{multirow}
\usepackage{setspace}
\usepackage{booktabs}


%constants
\newcommand{\E}{\mathrm{e}}
\newcommand{\P}{\mathrm{\pi}}
\newcommand{\im}{\mathrm{i}}
\newcommand{\hbar}{\bar{h}}
\newcommand{\rB}{\sqrt{2}}
\newcommand{\rC}{\sqrt{3}}

%units
\newcommand{\du}{^{\circ}}
\newcommand{\Rad}{\,\textrm{rad}}
\newcommand{\rps}{\,\mathrm{rad}/\mathrm{s}}

\newcommand{\m}{\,\textrm{m}}
\newcommand{\cm}{\,\textrm{cm}}
\newcommand{\mm}{\,\textrm{mm}}
\newcommand{\mum}{\,\mathrm{\mu m}}
\newcommand{\nm}{\,\textrm{nm}}
\newcommand{\sqm}{\,\mathrm{m}^2}

\newcommand{\s}{\,\textrm{s}}
\newcommand{\ms}{\,\mathrm{ms}}
\newcommand{\mus}{\,\mathrm{\mu s}}
\newcommand{\mps}{\,\mathrm{m/s}}
\newcommand{\mpss}{\,\textrm{m/s}^{2}}
\newcommand{\hz}{\,\textrm{Hz}}
\newcommand{\khz}{\,\textrm{kHz}}
\newcommand{\mhz}{\,\textrm{MHz}}

\newcommand{\g}{\,\textrm{g}}
\newcommand{\kg}{\,\textrm{kg}}
\newcommand{\gmm}{\,\textrm{g}\cdot\textrm{m}^2}
\newcommand{\N}{\,\mathrm{N}}
\newcommand{\pa}{\,\textrm{Pa}}
\newcommand{\kpa}{\,\mathrm{kPa}}
\newcommand{\mmhg}{\,\mathrm{mmHg}}

\newcommand{\J}{\,\mathrm{J}}
\newcommand{\ev}{\,\mathrm{eV}}
\newcommand{\W}{\,\mathrm{W}}

\newcommand{\V}{\,\textrm{V}}
\newcommand{\kv}{\,\mathrm{kV}}
\newcommand{\mv}{\,\textrm{mV}}
\newcommand{\muv}{\,\mathrm{\mu V}}
\newcommand{\A}{\,\textrm{A}}
\newcommand{\ma}{\,\textrm{mA}}
\newcommand{\mua}{\,\mathrm{\mu A}}
\newcommand{\T}{\,\textrm{T}}
\newcommand{\mt}{\,\textrm{mT}}
\newcommand{\apm}{\,\mathrm{A}/\mathrm{m}}
\newcommand{\Apm}{\,(\mathrm{A}/\mathrm{m})}

\newcommand{\muf}{\,\mathrm{\mu F}}
\newcommand{\pf}{\,\textrm{pF}}
\newcommand{\om}{\,\Omega}
\newcommand{\ko}{\,\textrm{k}\Omega}
\newcommand{\mo}{\,\textrm{M}\Omega}
\newcommand{\mio}{\,\mathrm{m}\Omega}
\newcommand{\hen}{\,\textrm{H}}
\newcommand{\mh}{\,\textrm{mH}}
\newcommand{\muh}{\,\mathrm{\mu H}}

\newcommand{\cel}{\,^{\circ}\mathrm{C}}
\newcommand{\K}{\,\mathrm{K}}
\newcommand{\mk}{\,\mathrm{mK}}
\newcommand{\wpmk}{\,\mathrm{W}/(\mathrm{m}\cdot \mathrm{K})}


%blocks
\newcommand{\bl}{\big(}
\newcommand{\br}{\big)}
\newcommand{\bbl}{\bigg(}
\newcommand{\bbr}{\bigg)}
\newcommand{\Bl}{\Big(}
\newcommand{\Br}{\Big)}
\newcommand{\bL}{\big[}
\newcommand{\bR}{\big]}
\newcommand{\bbL}{\bigg[}
\newcommand{\bbR}{\bigg]}
\newcommand{\BL}{\Big[}
\newcommand{\BR}{\Big]}

\newcommand{\cbrak}[1]{\{ #1 \}}
\newcommand{\BraK}[1]{\BL{#1}\BR}
\newcommand{\Brak}[1]{\Bl{#1}\Br}
\newcommand{\expc}[1]{\langle{#1}\rangle}
\newcommand{\abs}[1]{\lvert{#1}\rvert}
\newcommand{\Abs}[1]{\Big\lvert{#1}\Big\rvert}
\newcommand{\ol}{\overline}
\newcommand{\ket}[1]{\lvert{#1}\rangle}
\newcommand{\bra}[1]{\langle{#1}\rvert}

%operators
\newcommand{\sins}{\sin^{2}}
\newcommand{\coss}{\cos^{2}}
\newcommand{\et}{\times 10^}
\newcommand{\eet}{10^}

\newcommand{\D}{\mathrm{d}}
\newcommand{\der}[2]{\dfrac{\textrm{d}{#1}}{\textrm{d}{#2}}}
\newcommand{\ptd}[2]{\dfrac{\partial{#1}}{\partial{#2}}}

\newcommand{\Sum}[4]{\sum_{#2=#3}^{#4}#1}
\newcommand{\Prod}[4]{\prod_{#2=#3}^{#4}#1}
\newcommand{\Int}[2]{\int#1\,\mathrm{d}#2}
\newcommand{\Intg}[4]{\int_{#3}^{#4}#1\,\mathrm{d}#2}

%formulas
\newcommand{\rssB}[2]{\sqrt{#1^2+#2^2}}
\newcommand{\rssC}[3]{\sqrt{#1^2+#2^2+#3^2}}

\newcommand{\lsmslope}[2]{\dfrac{\ol{{#2}{#1}}-\ol{#2}\,\ol{#1}}{\ol{{#2}^{2}}-\ol{#2}^{2}}}
\newcommand{\lsminterc}[3]{\ol{#1}-#3\ol{#2}}
\newcommand{\lsmcorr}[2]{\dfrac{\ol{{#2}{#1}}-\ol{#2}\,\ol{#1}}{\sqrt{(\ol{{#2}^{2}}-\ol{#2}^{2})(\ol{{#1}^{2}}-\ol{#1}^{2})}}}
\newcommand{\lsmunc}[3]{\sqrt{\dfrac{1/{#2}^{2}-1}{{#3}-2}}{#1}}
\newcommand{\lsmUnc}[3]{\dfrac{\sg{#2}}{\sqrt{\sum_{i=1}^{#1}(#3_i-\ol{#3})^2}}}
\newcommand{\lsmrmse}[5]{\sqrt{\dfrac{\sum_{i=1}^{#5}(#1_i-#3-#4#2_i)^2}{#5-2}}}

\newcommand{\stddev}[2]{\sqrt{\dfrac{\sum_{i=1}^{#2}({#1}_i-\ol{#1})}{#2-1}}}

%others
\newcommand{\sg}{\sigma_}
\newcommand{\ans}[1]{{#1}\pm\sg{#1}}

\newcommand{\sct}{\section}
\newcommand{\ssct}{\subsection}
\newcommand{\sssct}{\subsubsection}
\newcommand{\ssssct}{\subsubsubsection}

%text
\newcommand{\sj}{\quad\!\!\!\!\:}
\newcommand{\eA}{\:\!}
\newcommand{\eB}{\;\!}
\newcommand{\eF}{\,\,}
\newcommand{\eG}{\,\:}
\newcommand{\eH}{\,\;}
\newcommand{\ed}{\:\;}
\newcommand{\eJ}{\;\;}
\newcommand{\eK}{\,\,\;}
\newcommand{\eL}{\,\:\;}
\newcommand{\eM}{\:\:\;}
\newcommand{\eN}{\:\;\;}
\newcommand{\eO}{\;\;\;}
\newcommand{\eP}{\:\:\:\:}
\newcommand{\eT}{\;\;\;\;}
\newcommand{\eW}{\;\;\;\;\,}
\newcommand{\edd}{\:\;\:\;}
\newcommand{\eddd}{\:\;\:\;\:\;}
\newcommand{\bg}{\,Table\,\,}
\newcommand{\tp}{\,Figure\,\,}
\newcommand{\xh}[1]{\sj\textbf{(#1)}\edd}

%environments
\newcommand{\ceq}[1]{\begin{center}{#1}\end{center}}
\newcommand{\alg}[1]{\begin{align*}{#1}\end{align*}}
\newcommand{\ltb}[3]{\begin{longtable}{#1}
\caption{#2}
\centering
#3
\end{longtable}}
\newcommand{\fg}[3]{\begin{figure}[h]
\centering
\includegraphics[scale=#1]{#2}
\caption{#3}
\end{figure}}
\newcommand{\zt}[1]{\mathrm{#1}}

%temporary
\newcounter{RomanNumber}
\newcommand{\rom}[1]{\setcounter{RomanNumber}{#1}\Roman{RomanNumber}}
\newcommand{\lc}{L_{\zt{c}}}
\newcommand{\lt}{L_{\zt{t}}}
\newcommand{\qm}{\cm^2}


\pagestyle{fancy}
\setlength{\textwidth}{159.2mm}
\setlength{\oddsidemargin}{0pt}
\setlength{\voffset}{-10mm}
\setlength{\headwidth}{159.2mm}
\setlength{\textheight}{235mm}
\fancyhf{}
\lhead{\begin{CJK*}{UTF8}{gkai}实验二十七\ 交流电桥\end{CJK*}}
\rhead{\thepage}

\begin{document}
\begin{CJK*}{UTF8}{gbsn}

\title{\textbf{实验二十七\ 交流电桥\\实验报告}}
\author{葛博文\;1500019707\\周二\,7\,组\,1\,号}
\date{2018\,年\,6\,月\,8\,日}
\maketitle

\begin{spacing}{1.2}
\setcounter{section}{1}
\sct*{一\;\;\;\;数据处理}
\ssct{电容桥测量电容}
\sssct{实验数据}
\sj使用元件盒,纸质电容标称值$\,C_1=0.2\muf$,电解电容标称值$\,C_2=6.8\muf$。信号发生器频率$\,f=1.000\,1\khz$,电压有效值$\,U=4.015\V$,波形为正弦波。数据记录在\bg1\,中。
\ltb{|c|c|c|c|c|c|c|c|c|}{用电容桥测量电容实验数据}{
\hline
& $R_1/\om$ & $R_2/\om$ & $R_0/\om$ & $C_0/\muf$ & $\Delta U/\mv$ & $C_x/\muf$ & $R_C/\om$ & $\tan\delta$ \\ \hline
纸质电容 & \ed\,500.0 & \ed\,500.0 & \ed2.4 & 0.227\,6 & 0.15 & 0.227\,6 & 2.40\ed & 0.003\,36\\
电解电容 & \ed\,100.0 & 1\,000.0 & 35.0 & 0.680\,0 & 0.02 & 6.800\,\ed & 3.500 & 0.149\,5\ed\\
\hline
}
\sssct{纸质电容误差分析}
\sj两桥臂电阻$\,R_1,R_2\,$的不确定度为
\alg{
\sg{R_1}=\sg{R_2}=(500\om\times0.1\%)/\sqrt{3}+12\mio=0.3\om\\
}
\noindent因$\,R_0\,$太小,须看作离散变化的,故估计其不确定度为
\alg{
\sg{R_0}=0.1\om/\rC+12\mio=0.07\om
}
\noindent电容箱电容值的不确定度为
\alg{
\sg{C_0}=(0.2\muf\times0.5\%+0.02\muf\times0.65\%+0.007\muf\times2\%+0.0006\muf\times2\%)/\rC=7\et{-4}\muf=
}
\noindent实验过程中信号发生器频率有$\,3\hz\,$左右的变化,估计频率的不确定度为$\,\sg{f}=3\hz/\rC=1.7\hz$。
\noindent可以得出待测电容$\,C_x$、损耗电阻$\,R_C\,$和损耗$\,t=\tan\delta\,$的不确定度
\alg{
\sg{C_x}=&\rssC{\Brak{\dfrac{C_x}{R_1}\sg{R_1}}}{\Brak{\dfrac{C_x}{R_2}\sg{R_2}}}{\Brak{\dfrac{C_x}{C_0}\sg{C_0}}}=7\et{-4}\muf\\[8pt]
\sg{R_C}=&\rssC{\Brak{\dfrac{R_C}{R_1}\sg{R_1}}}{\Brak{\dfrac{R_C}{R_2}\sg{R_2}}}{\Brak{\dfrac{R_C}{R_0}\sg{R_0}}}=0.07\om\\[8pt]
\sg{t}=&\rssC{\Brak{\dfrac{\tan\delta}{R_0}\sg{R_0}}}{\Brak{\dfrac{\tan\delta}{C_0}\sg{C_0}}}{\Brak{\dfrac{\tan\delta}{f}\sg{f}}}=8\et{-5}
}
\par
\sj最终结果为
\alg{
\ans{C_x}&=(0.223\,0\pm0.000\,7)\muf\\[3pt]
\ans{R_C}&=(2.40\pm0.07)\om\\[3pt]
\tan\delta\pm\sg{t}&=(3.36\pm0.08)\et{-3}
}
\sssct{电解电容误差分析}
\sj两桥臂电阻$\,R_1,R_2\,$的不确定度为
\alg{
\sg{R_1}=(100\om\times0.1\%)/\sqrt{3}+12\mio=0.07\om\\[3pt]
\sg{R_2}=(1000\om\times0.1\%)/\sqrt{3}+12\mio=0.6\om
}
\noindent因$\,R_0\,$太小,须看作离散变化的,故估计其不确定度为
\alg{
\sg{R_0}=0.1\om/\rC+12\mio=0.07\om
}
\noindent电容箱电容值的不确定度为
\alg{
\sg{C_0}=(0.6\muf\times0.5\%+0.08\muf\times0.65\%)/\rC=2.0\et{-3}\muf
}
\noindent实验过程中信号发生器频率有$\,3\hz\,$左右的变化,估计频率的不确定度为$\,\sg{f}=3\hz/\rC=1.7\hz$。\\
\noindent可以得出待测电容$\,C_x$、损耗电阻$\,R_C\,$和损耗$\,t=\tan\delta\,$的不确定度
\alg{
\sg{C_x}=&\rssC{\Brak{\dfrac{C_x}{R_1}\sg{R_1}}}{\Brak{\dfrac{C_x}{R_2}\sg{R_2}}}{\Brak{\dfrac{C_x}{C_0}\sg{C_0}}}=0.021\muf\\[8pt]
\sg{R_C}=&\rssC{\Brak{\dfrac{R_C}{R_1}\sg{R_1}}}{\Brak{\dfrac{R_C}{R_2}\sg{R_2}}}{\Brak{\dfrac{R_C}{R_0}\sg{R_0}}}=0.008\om\\[8pt]
\sg{t}=&\rssC{\Brak{\dfrac{\tan\delta}{R_0}\sg{R_0}}}{\Brak{\dfrac{\tan\delta}{C_0}\sg{C_0}}}{\Brak{\dfrac{\tan\delta}{f}\sg{f}}}=6\et{-4}
}
\par
\sj最终结果为
\alg{
\ans{C_x}&=(6.800\pm0.021)\muf\\[3pt]
\ans{R_C}&=(3.500\pm0.008)\om\\[3pt]
\tan\delta\pm\sg{t}&=0.149\,5\pm0.000\,6
}


\ssct{用麦克斯韦-维恩桥和麦克斯韦桥分别测量同一无铁芯电感}
\sssct{实验数据}
\sj使用$\,\langle12\rangle\,$号元件盒,电感标称值$\,L=10\mh$,直流电阻标称值$\,R_L=100\om$。信号发生器频率$\,f=1.000\,9\khz$,电压有效值$\,U=4.015\V$,波形为正弦波。电感箱电感为$\,5\mh\,$时直流电阻为$\,R_{L_0}=3.92\om$。数据记录在\bg2\,和\bg3\,中。
\ltb{|c|c|c|c|c|c|c|c|}{用麦克斯韦-维恩桥测量电感}{
\hline
$R_1/\om$ & $R_2/\om$ & $R_0/\om$ & $\,C_0/\muf\,$ & $\Delta U/\mv$ & $L_x/\mh$ & $R_L/\om$ & $Q$ \\ \hline
150.0 & 150.0 & 213.0 & 0.440\,9 & 0.02 & 9.92 & 105.63 & 0.590\,2\\ \hline
}
\ltb{|c|c|c|c|c|c|c|c|}{用麦克斯韦桥测量电感}{
\hline
$R_1/\om$ & $R_2/\om$ & $R_0/\om$ & $L_0/\mh$ & $\Delta U/\mv$ & $L_x/\mh$ & $R_L/\om$ & $Q$ \\ \hline
500.0 & 250.5 & 49.0 & 5 & 0.67 & 9.98 & 105.63 & \eF0.594\eF\\ \hline
}
\sssct{麦克斯韦-维恩桥误差分析}
\sj两桥臂电阻$\,R_1,R_2\,$的不确定度为
\alg{
\sg{R_1}=\sg{R_2}=(150\om\times0.1\%)/\sqrt{3}+12\mio=0.09\om\\
}
\noindent$R_0\,$不确定度为
\alg{
\sg{R_0}=(210\om\times0.1\%+3\om\times0.5\%)/\rC+12\mio=0.13\om
}
\noindent电容箱电容值的不确定度为
\alg{
\sg{C_0}=(0.4\muf\times0.5\%+0.04\muf\times0.65\%+0.000\,9\muf\times5\%)/\rC=1.3\et{-3}\muf
}
\noindent实验过程中信号发生器频率有$\,3\hz\,$左右的变化,估计频率的不确定度为$\,\sg{f}=3\hz/\rC=1.7\hz$。
\noindent可以得出待测电感$\,L_x$、损耗电阻$\,R_L\,$和品质因数$\,Q\,$的不确定度
\alg{
\sg{L_x}=&\rssC{\Brak{\dfrac{L_x}{R_1}\sg{R_1}}}{\Brak{\dfrac{L_x}{R_2}\sg{R_2}}}{\Brak{\dfrac{L_x}{C_0}\sg{C_0}}}=0.03\mh\\[8pt]
\sg{R_L}=&\rssC{\Brak{\dfrac{R_L}{R_1}\sg{R_1}}}{\Brak{\dfrac{R_L}{R_2}\sg{R_2}}}{\Brak{\dfrac{R_L}{R_0}\sg{R_0}}}=0.11\om\\[8pt]
\sg{Q}=&\rssC{\Brak{\dfrac{Q}{R_0}\sg{R_0}}}{\Brak{\dfrac{Q}{C_0}\sg{C_0}}}{\Brak{\dfrac{Q}{f}\sg{f}}}=2.0\et{-3}
}
\par
\sj最终结果为
\alg{
\ans{L_x}&=(9.92\pm0.03)\mh\\[3pt]
\ans{R_L}&=(105.63\pm0.11)\om\\[3pt]
\ans{Q}&=0.590\,2\pm0.002\,0
}
\sssct{麦克斯韦桥误差分析}
\sj两桥臂电阻$\,R_1,R_2\,$的不确定度为
\alg{
\sg{R_1}=&(500\om\times0.1\%)/\sqrt{3}+12\mio=0.3\om\\[3pt]
\sg{R_2}=&(250\om\times0.1\%+0.5\om\times2\%)/\sqrt{3}+12\mio=0.16\om
}
\noindent因$\,R_0\,$太小,须看作离散变化的,故估计其不确定度为
\alg{
\sg{R_0}=0.1\om/\rC+12\mio=0.07\om
}
\noindent电感箱电容值的不确定度为
\alg{
\sg{L_0}=(5\mh\times2\%)/\rC=0.06\mh
}
\noindent实验过程中信号发生器频率有$\,3\hz\,$左右的变化,估计频率的不确定度为$\,\sg{f}=3\hz/\rC=1.7\hz$。
\noindent可以得出待测电感$\,L_x$、损耗电阻$\,R_L\,$和品质因数$\,Q\,$的不确定度
\alg{
\sg{L_x}=&\rssC{\Brak{\dfrac{L_x}{R_1}\sg{R_1}}}{\Brak{\dfrac{L_x}{R_2}\sg{R_2}}}{\Brak{\dfrac{L_x}{L_0}\sg{L_0}}}=0.12\mh\\[8pt]
\sg{R_L}=&\rssC{\Brak{\dfrac{R_L}{R_1}\sg{R_1}}}{\Brak{\dfrac{R_L}{R_2}\sg{R_2}}}{\Brak{\dfrac{R_L}{R_0+R_{L_0}}\sg{R_0}}}=0.17\om\\[8pt]
\sg{Q}=&\rssC{\Brak{\dfrac{Q}{R_0+R_{L_0}}\sg{R_0}}}{\Brak{\dfrac{Q}{L_0}\sg{L_0}}}{\Brak{\dfrac{Q}{f}\sg{f}}}=7\et{-3}
}
\par
\sj最终结果为
\alg{
\ans{L_x}&=(9.98\pm0.12)\mh\\[3pt]
\ans{R_L}&=(105.63\pm0.17)\om\\[3pt]
\ans{Q}&=0.594\pm0.007
}
\sssct{比较两种电桥的收敛性}
\sj实验中麦克斯韦-维恩桥的收敛性好于麦克斯韦桥。麦克斯韦-维恩桥只调节了一次$\,R_0\,$和一次$\,C_0$,就使桥路电压降至$\,0.02\mv$,收敛得很快;而麦克斯韦桥需要来回调节三次$\,R_0\,$和$\,R_2\,$才使桥路电压降至$\,0.67\mv$,且无法进一步下降,其收敛性较差。


\ssct{测量标准互感器的互感}
\sssct{实验数据}
\sj互感标称值为$\,M=0.05\hen$,用万用表测得线圈\,I\,直流电阻为$\,R_{\mathrm{I}}=53.20\om$,线圈\,II\,的直流电阻为$\,R_{\mathrm{II}}=56.54\om$,两线圈间为断路。信号发生器频率为$\,f=1.000\,5\khz$,电压有效值为$\,U=4.005\V$,波形为正弦波。采用麦克斯韦-维恩桥分别测量互感器顺接($\mathrm{I}_1$\,端接\,$\mathrm{II}_2$\,端)和反接($\mathrm{I}_1$\,端接\,$\mathrm{II}_1$\,端)时的电感$\,L_{\zt{c}}\,$和$\,L_{\zt{t}}$,数据记录在\bg4\,中。
\ltb{|c|c|c|c|c|c|c|c|}{测量互感器顺接和反接时的电感}{
\hline
接法 & $R_1/\om$ & $R_2/\om$ & $R_0/\om$ & $C_0/\muf$ & $\Delta U/\mv$ & $L_x/\mh$ & $R_L/\om$  \\ \hline
顺接 & 600.0 & 600.0 & 3\,301.2 & 0.700\,3 & 0.14 & 252.1\ed & 109.1\\
反接 & 300.0 & 300.0 & \ed\,823.6 & 0.589\,4 & 0.04 & \ed53.05 & 109.3\\ \hline
}\par
\sj计算出互感器的互感为$\,M=(\lc-\lt)/4=49.76\mh$。
\sssct{误差分析}
\sj顺接时,两桥臂电阻$\,R_1,R_2\,$的不确定度为
\alg{
\sg{R_1}=\sg{R_2}=(600\om\times0.1\%)/\sqrt{3}+12\mio=0.4\om\\
}
\noindent$R_0\,$不确定度为
\alg{
\sg{R_0}=(3\,300\om\times0.1\%+1\om\times0.5\%+0.2\om\times2\%)/\rC+12\mio=1.9\om
}
\noindent电容箱电容值的不确定度为
\alg{
\sg{C_0}=(0.7\muf\times0.5\%+0.000\,3\muf\times5\%)/\rC=2.0\et{-3}\muf
}
\noindent实验过程中信号发生器频率有$\,3\hz\,$左右的变化,估计频率的不确定度为$\,\sg{f}=3\hz/\rC=1.7\hz$。
\noindent可以得出顺接电感$\,\lc\,$的不确定度
\alg{
\sg{\lc}=&\rssC{\Brak{\dfrac{\lc}{R_1}\sg{R_1}}}{\Brak{\dfrac{\lc}{R_2}\sg{R_2}}}{\Brak{\dfrac{\lc}{C_0}\sg{C_0}}}=0.8\mh
}
\par
\sj反接时,两桥臂电阻$\,R_1,R_2\,$的不确定度为
\alg{
\sg{R_1}=\sg{R_2}=(300\om\times0.1\%)/\sqrt{3}+12\mio=0.18\om\\
}
\noindent$R_0\,$不确定度为
\alg{
\sg{R_0}=(820\om\times0.1\%+3\om\times0.5\%+0.6\om\times2\%)/\rC+12\mio=0.5\om
}
\noindent电容箱电容值的不确定度为
\alg{
\sg{C_0}=&(0.5\muf\times0.5\%+0.08\muf\times0.65\%+0.009\muf\times2\%+0.000\,4\muf\times5\%)/\rC\\[3pt]
=&1.9\et{-3}\muf
}
\noindent实验过程中信号发生器频率有$\,3\hz\,$左右的变化,估计频率的不确定度为$\,\sg{f}=3\hz/\rC=1.7\hz$。
\noindent可以得出反接电感$\,\lt\,$的不确定度
\alg{
\sg{\lt}=&\rssC{\Brak{\dfrac{\lt}{R_1}\sg{R_1}}}{\Brak{\dfrac{\lt}{R_2}\sg{R_2}}}{\Brak{\dfrac{\lt}{C_0}\sg{C_0}}}=0.18\mh
}
\par
\sj互感的不确定度为
\alg{
\sg{M}=\dfrac{1}{4}\rssB{\sg{\lc}}{\sg{\lt}}=0.21\mh
}
\par
\sj最终结果为
\alg{
\ans{M}=(49.76\pm0.21)\mh
}


\ssct{用麦克斯韦-维恩桥测磁环不同频率下的电感、损耗电阻和相对磁导率}
\sj使用$\,\langle2\rangle\,$号磁环,待测磁环匝数$\,N=180$,平均直径$\,\ol{D}=8.56\cm$,横截面积$\,S=2.00\qm$,等效磁路长度$\,l=\pi\ol{D}=26.9\cm$。信号发生器电压有效值为$\,U=1.005\,0\V$,波形为正弦波。用麦克斯韦-维恩桥测出磁环的电感$\,L$,损耗电阻$\,R_L$,并根据$\,L=\frac{\mu\mu_0N^2S}{\pi\ol{D}}\,$计算出相对磁导率$\,\mu=\frac{\pi\ol{D}L}{\mu_0N^2S}$。实验数据记录在\bg5\,中。
\ltb{|c|c|c|c|c|c|c|c|c|}{测磁环不同频率下的电感、损耗电阻和相对磁导率}{
\hline
$f/\khz$ & $R_1/\om$ & $R_2/\om$ & $R_0/\om$ & $C_0/\muf$ & $\Delta U/\mv$ & $L/\mh$ & $R_L/\om$ & $\mu$ \\
\hline
\endhead

\hline
\ed0.099\,97 & 100.0 & 100.0 & 6\,830.5 & 0.110\,0 & 0.03 & 1.100\eL & 1.464\,0 & 36.3\ed \\
\ed0.399\,82 & 100.0 & 100.0 & 4\,530.6 & 0.053\,6 & 0.08 & 0.536\eL & 2.207\,2 & 17.70 \\
\ed0.700\,12 & \ed80.0 & \ed80.0 & 2\,400.6 & 0.061\,6 & 0.02 & 0.394\eL & 2.666\eL & 13.01 \\
\ed1.001\,4\ed & \ed80.0 & \ed80.0 & 2\,120.6 & 0.050\,1 & 0.02 & 0.326\eL & 3.018\eL & 10.77 \\
\ed2.001\,1\ed & \ed80.0 & \ed80.0 & 1\,650.0 & 0.034\,1 & 0.03 & 0.218\,3 & 3.879\eL & \ed7.21 \\
\ed3.001\,0\ed & \ed80.0 & \ed80.0 & 1\,425.0 & 0.027\,3 & 0.01 & 0.174\,7 & 4.491\eL & \ed5.77 \\
\ed5.002\,2\ed & \ed80.0 & \ed80.0 & 1\,161.9 & 0.020\,9 & 0.10 & 0.133\,8 & 5.508\eL & \ed4.42 \\
\ed7.004\,0\ed & \ed80.0 & \ed80.0 & 1\,028.0 & 0.017\,2 & 0.03 & 0.110\,1 & 6.226\eL & \ed3.64 \\
\ed9.005\,0\ed & \ed80.0 & \ed80.0 & \eL920.0 & 0.015\,6 & 0.06 & 0.099\,8 & 6.957\eL & \ed3.30 \\
10.014\ed\eL & \ed80.0 & \ed80.0 & \eL890.0 & 0.015\,0 & 0.02 & 0.096\,0 & 7.191\eL & \ed3.17 \\
\hline
}
\par
\sj作出$\,L-f$,$R_L-f\,$和$\,\mu-f\,$图如第\,10\,页\tp$3-5$。\newpage





\setcounter{section}{2}
\setcounter{subsection}{0}
\sct*{二\eT思考题}
\sj麦克斯韦-维恩桥平衡时有
\alg{
\dfrac{1}{R_0}+\im\omega C_0=\dfrac{R_L}{R_1R_2}+\im\dfrac{\omega L_x}{R_1R_2}
}
\par
\sj记等号左边为复数$\,B$,右边为复数$\,A$。实验中固定$\,A$,分别调节$\,R_0\,$和$\,C_0$,即独立地改变$\,B\,$的实部和虚部,使$\,B\,$尽量靠近$\,A$,从而使电桥平衡。\par
\sj如\tp1(a)\,所示,调节$\,R_0\,$改变$\,B\,$的实部,$B\,$在复平面上沿直线$\,ef\,$变化;调节$\,C_0\,$改变$\,B\,$的虚部,$B\,$在复平面上沿直线$\,cd\,$变化。实验中可以先调节$\,R_0\,$使电压表示数最小,再调节$\,C_0\,$使电压表示数最小,相当于\tp1(b)\,中$\,B\,$先移动到$\,g_1\,$点再移动到$\,A$;也可以先调节$\,C_0\,$使电压表示数最小,再调节$\,R_0\,$使电压表示数最小,相当于\tp1(b)\,中$\,B\,$先移动到$\,g_2\,$点再移动到$\,A$。
\fg{1}{4.jpg}{电桥达到平衡的过程图}

\setcounter{section}{3}
\setcounter{subsection}{0}
\sct*{三\eT分析与讨论}
\ssct{实验中元件参数的选择对测量结果的精度有什么影响?}
\sj实验中元件参数的选择对测量结果的影响有如下几方面:
\begin{itemize}
\item 若桥臂电阻$\,R_1,R_2\,$选择得过大或过小,超出一定范围,可能会导致根本无法将电桥调至平衡;
\item 若桥臂电阻$\,R_1,R_2\,$选择得过大,会使电桥的灵敏度下降,测量结果精度降低,且电阻箱各挡间分布电容接入电路会带来额外的误差;
\item 若桥臂电阻$\,R_1,R_2\,$选择得过小,在$\,50\om\,$以内,会增大电阻箱本身的相对误差,从而使测量结果精度降低。
\end{itemize}

\ssct{实验中两种电桥测量电感的收敛性差别是否很大,与什么因素有关?}
\sj实验中麦克斯韦-维恩桥和麦克斯韦桥的收敛性有明显差别,麦克斯韦-维恩桥只需调节一次就可使电桥平衡,但是麦克斯韦桥需要调三次才能平衡。原因在于麦克斯韦-维恩桥在调平衡的过程中独立地改变复导纳的实部和虚部,故只需调节一次电容一次电阻即可平衡(见第二部分);而麦克斯韦桥需要不断调节$\,R_0\,$和$\,R_2\,$去逼近(见\tp2),无法直接调平衡,收敛性较差。\par
\fg{0.8}{5}{麦克斯韦桥调平衡的过程}
\sj麦克斯韦桥平衡时有
\alg{
R_2(R_L+\im\omega L_x)=R_1(R_0+R_{L_0}+\im\omega L_0)
}
\noindent将等式左边记为复数$\,A$,等式右边记为复数$\,B$,调节平衡的过程如\tp2\,所示。在\tp2\,中可直观地看出,麦克斯韦桥的收敛性是和$\,eb\,$与$\,Ob\,$的夹角(即$\,A\,$的辐角)有关的,这个夹角越大收敛得越快,夹角越小收敛得越慢。故$\,A\,$的辐角越大,麦克斯韦桥的收敛性越好。$A\,$的辐角为$\,\arg{A}=\frac{\omega L_x}{R_L}=Q$,即待测电感的$\,Q\,$值越大,麦克斯韦桥的收敛性越好。

\ssct{磁环的损耗电阻和磁导率随频率怎样变化?为什么?}
\sj磁环的损耗电阻随频率升高而增大,因频率升高时铁芯中涡流损耗会显著增大,在电路中表现为损耗电阻增大。\par
\sj磁环的磁导率频率升高而减小,因为有磁滞现象,高频信号下,铁芯的磁化跟不上外加励磁电流的变化,故磁化的程度减弱,表现为磁导率降低。


\setcounter{section}{4}
\setcounter{subsection}{0}
\sct*{四\eT收获与感想}
\sj本实验中元件参数的选择十分重要,会影响测量结果的精度,甚至对实验是否成功有决定性影响。在测量互感的实验中,开始时我将桥臂电阻调得过大,以至于在电容箱和电阻箱的调节范围内无法将电桥调平,导致实验失败;在认识到桥臂电阻过大后,我将其调整到合适的数值,便可以将电桥调平了,从而认识到了元件参数选择的重要性。

\fg{0.35}{1.JPG}{$L-f\,$关系图}
\fg{0.35}{2.JPG}{$R_L-f\,$关系图}
\fg{0.35}{3.JPG}{$\mu-f\,$关系图}



\end{spacing}
\end{CJK*}
\end{document}