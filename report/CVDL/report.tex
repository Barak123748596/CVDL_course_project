\documentclass[a4paper,12pt]{article}
\usepackage{CJKutf8}
\usepackage{indentfirst}
\usepackage{graphicx}
\usepackage{amsmath}
\usepackage{longtable}
\usepackage{fancyhdr}
\usepackage{multirow}
\usepackage{setspace}
\usepackage{booktabs}
\usepackage{daytime}

\renewcommand{\today}{\number\year 年 \number\month 月 \number\day 日}

\pagestyle{fancy}
\setlength{\textwidth}{159.2mm}
\setlength{\oddsidemargin}{0pt}
\setlength{\voffset}{-10mm}
\setlength{\headwidth}{159.2mm}
\setlength{\textheight}{235mm}
\fancyhf{}
\lhead{\begin{CJK*}{UTF8}{gkai}CVDL大作业\end{CJK*}}
\rhead{\thepage}

\begin{document}
\begin{CJK*}{UTF8}{gbsn}

\title{\textbf{CVDL大作业\\期末报告}}
\author{葛博文\;1500019707\\周清逸\;1500012930}
\date{\today}
\maketitle

\begin{spacing}{1.2}
    \section{任务动机}
    在一次寒假调研时,我们意外地发现,尽管地方政府对农业用地早已有了诸多保护政策,农业用地违法挪用建造商品房以谋取私利的情况依然广大范围内存在。这种行为对我国的农业生产工作产生了极大的危害,必须严加监管。而这现象屡禁不止的原因之一,是高层政府对于农业用地的实际监管困难。因此我们希望做一个程序来试图识别农业用地的实际使用情况。
    \section{任务分析}
    如何从遥感卫星图片判断一块土地是否为农业用地是一个比较泛泛的概念。考虑到很多土地挪用是为了商品房建造,同时房屋的数据集也比较容易得到,因此我们可以将这个土地使用方式识别的任务转化为在一片地图上寻找房屋的任务。我们从“一个网站”上得到了有关房屋的数据集。初步分析数据集我们可以猜想这个任务的主要难点在于:
    \begin{itemize}
        \item 房子数量多而分布密集,单个房子却很小,会对网络产生压力。
        \item 由于光照问题,房屋和树木遮挡,会相应地产生阴影投射到房子上,对于网络的识别增加了困难。
        \item 不同数据集之间差别很大,而且不均匀。例如有些城市沿海,就会有少量的水面和船等景象,然而这些数量很小,不能在网络中充分训练。
        \item 房屋和船、大卡车等物品天生很像。
        \item 在这些数据集上有关绿地的部分主要为树木和草地,而作为我们最终目标的中国地区作为测试的图片中,却主要为农田,这样的数据集变化可能会导致很糟糕的效果。
    \end{itemize}
\end{spacing}
\end{CJK*}
\end{document}